\documentclass[12pt,a4paper]{article}

\usepackage[UTF8]{ctex}
\usepackage{microtype}
\usepackage{etoolbox}
\AtEndPreamble{\AtBeginDocument{\microtypesetup{disable}{\em}\microtypesetup{enable}}}

\usepackage{hyphenat}
\usepackage{fontspec}

\newfontfamily\promptfont{ Open Sans }[%
    SizeFeatures={Size=9},
    Numbers={Proportional, OldStyle}
]

\newfontfamily\titlefont{ Open Sans }[%
    SizeFeatures={Size=12},
    Numbers={Proportional, OldStyle}
]

\newfontfamily\bodyfont{ EB Garamond }[%
    SizeFeatures={Size=12},
    Numbers={Proportional, OldStyle}
]

\setCJKmainfont{ Noto Serif SC }[SizeFeatures={Size=11}]

\renewcommand{\baselinestretch}{1.3}
\raggedbottom

\newcommand{\dashpad}{\hspace{0em plus 0pt}}
\newcommand{\mdash}{\dashpad\textemdash\dashpad}
\newcommand{\ndash}{\dashpad\textendash\dashpad}
\newcommand{\ellipsis}{\kern.12em\textellipsis\hspace{.12em plus 1pt}}

\newcommand{\separator}{
    \begin{center}
        *\quad * \quad *
    \end{center}
}

\usepackage[colorlinks=true]{hyperref}
\usepackage[framemethod=TikZ]{mdframed}
\usepackage{todonotes}
\usepackage{xcolor}
\definecolor{notecolor}{RGB}{193, 212, 245}
\definecolor{ntextcolor}{RGB}{235, 30, 75}

\newcommand{\note}[1]{\renewcommand{\baselinestretch}{1}%
    \todo[backgroundcolor=notecolor,linecolor=notecolor,bordercolor=notecolor]{\bodyfont #1}}

\newcommand{\ntext}[1]{\textcolor{ntextcolor}{ #1}}

\usepackage[%
    left=2.2cm,
    right=9.1cm,
    top=1.8cm,
    bottom=2.8cm,
    marginparwidth=7.2cm,
    marginparsep=.5cm
]{geometry}

\usepackage{parskip}


\usepackage{graphicx}
\usepackage[%
    contents={ \includegraphics[scale=.02]{ C:/Users/jshen/Documents/cfpdfmaker/assets/watermarks/zeyou.png } },
    opacity=1,
    angle=0,
]{background}


\begin{document}


{
    
    \promptfont
    
    \href{http://www.harvard.edu}{Harvard} has long recognized the importance of student body diversity of all kinds. We welcome you to write about distinctive aspects of your **background**, **personal development** or the **intellectual interests** you might bring to your Harvard classmates.
}

{
    
    \titlefont
    
    A Day Without Democracy
}

{
    
    \bodyfont
    
    Tiger, tiger \\Burning bright \\ In the dark forest \\of the night
}

{
    
    \bodyfont
    
    It was pitch black. I was struggling so badly that I could feel droplets of sweat appearing on my forehead. I had a stick in my hand, and I was trying to navigate my way through a maze. My only hope was following the voice of the blind guide but I seemed to have already lost him. An instant coldness filled my body, fear was paralyzing me. I was starting to understand what it meant to lose something you have taken for granted\mdash your sense of sight. Even though I was aware what I was getting into when I signed up to be a part of the dialogue in the dark I wasn't expecting the feeling of helplessness that came with it. I'd been volunteering at the Six Dots Foundation for the blind as a teacher for blind young adults, but that day was the first time I was able to start grasping their situations. \ntext{Only when you lose something do you truly understand its worth.}\note{一大段看似与主题无关,其实用心良苦,后文可见。但这样写也很冒险,可能包袱还来不及抖招生官就失去了耐心。}
}

{
    
    
}

{
    
    \bodyfont
    
    Would you rather lose your sense of sight, sense of hearing, or your freedom? \ntext{A day without democracy was similar to a day in the dark:}\note{以失明喻失去民主,转入正题。} July 15, 2016. It all started around 10 p.m. The bridges were shut down for the public and soldiers swarmed the streets. I was transfixed in front of the TV, confused, as if someone had turned off the lights permanently but hadn’t warned us of anything. A few hours later, members of the parliament were taken hostage. A council ironically named ``Peace at Home'' had taken over a news channel and proclaimed that a faction of the Turkish military had made a coup attempt against the government. A female newscaster was forced to read their proposal, her eyes were shining with tears, much like my own. I'd never been in a situation like this, but my parents had; \ntext{they were my blind guides in this darkness.}\note{继续利用失明的比喻。若不充分利用,第一段就是离谱的闲笔。} My mom began calmly counting the food in the fridge while my dad tried to explain the politics of the situation to me. I couldn't catch up; I couldn't register what they had understood so fast. I started feeling the same helplessness I felt when I was losing the tour guide. \ntext{I was hitting the walls with my stick, and the distance between me and my guide grew with each passing minute.}\note{多数人既没有经历过失明也没有经历过政变,但作者偏以此喻彼,反而很能调动读者的想象。}
}

{
    
    \bodyfont
    
    As I heard F-16s flying over my apartment, barely missing the roof, deafening sounds rang in my ears. I looked at my parents' terrified faces; they tried to tell me that the loud noises were coming from sound bombs that were intended to scare people off the streets. That day I faced the threats of not going to be able to fly back to school, of not being able to go outside but most importantly, of losing the democracy in my country. That day I saw people lose their lives for democracy.
}

{
    
    \bodyfont
    
    And then suddenly the lights were back on. The coup attempt had been repelled. At that moment I understood what it means to lose something I had taken for granted throughout my life: \ntext{my freedom.}\note{前文可以照应此处my freedom的内容无非是作者担心非常状态下出不了门,此外并没有表现出对民主和自由的关系有多深的理解。这一领悟来得有点勉强。}
}

{
    
    \bodyfont
    
    I learned to overcome fear by focusing on the difference I could make to improve the situation in Turkey. \ntext{I believe that only education and economic prosperity can beat terror.}\note{过渡生硬。前文完全看不出作者何以认为教育能对抗恐怖,读者难免有“因为这是申请文书所以这么写”的感觉。} As a result, I'd like to focus on social impact in my college education and work in the newly emerging field of Social Entrepreneurship in the future. \ntext{I would like to continue and grow my NGO,}\note{以下文字和前文全靠democracy这一关键词勉强维持着一点点联系。即使对于志在哈佛的高中生来说,政变之类的题材似乎还是太难把握。} the Give A Hand Organization, while also working to benefit the Cambridge Community. I can achieve this dream at Harvard through courses such as *Practicing Democracy*, where I can put theory into practice, or gain non-profit experience through numerous fellowships, such as Mindich Service Fellows Program. \ntext{As an aspiring international lawyer, the future Liman Public Interest Law Fellowship is a once-in-a-lifetime opportunity.}\note{这种话放在一般的why school里面毫无问题,放在这里则好像是从别的文书里拿来硬塞进来的。} I am motivated to work harder each day because my experience does not let me forget that there are people who live with the fear of war and unrest, and they do not have the power to change things the way I do.
}

{
    
    Diversity文书就是要百花齐放,任何深刻影响自己的经历都值得一书,如果是人无我有的经历就更好。当然政变可遇不可求,但作者的写作手法还是值得我们借鉴的:以失明为extended metaphor贯穿始终;只写亲见亲闻,即使天翻地覆之时作者的视角也始终不曾离开自己的屋子,使当时的迷茫和恐惧显得真实(若开启上帝视角写叛军的种种作为,就不会有心理上的真实感)。本文的缺点也很明显:作者想把亲历的历史和自己的学术兴趣联系起来,但对历史认识不深,学养不足(对高中生也确实不能要求太高),说不出个所以然来。
}


\end{document}